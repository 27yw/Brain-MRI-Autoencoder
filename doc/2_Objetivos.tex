\chapter{Scope}
\label{chapter:scope}

In this chapter, we will establish the aims of the project. We have just spoken about the problem to be solved. So now we have to enumerate the concrete objectives of the project. 

\TBD{The objectives will be temporal and will be redefined in further stages.}


\section{Hypothesis}

\textbf{We will build an auto encoder for reconstructing and denoise T1-weighted brain MRI. It will remove noise and will learn the underlying structure of the images in a lower dimensional space, and will reconstruct the image based on this low dimensional space.}

\section{Primary aims}

\begin{itemize}
    \item To build a \textbf{noise-reducer autoencoder} that gets good results with control \textbf{T1-weighted brain MRI}: given a T1-WBMRI, the autoencoder will return the same image as equal as we can to the original, but removing the noise.
    \item Research a good autoencoder architecture and parameters (loss function, batch-norm or not batch-norm, regularization, etc).
    \item Establish a good brain MRI pre-processing.
\end{itemize}

\section{Secondary aims}

\begin{itemize}
    \item Develop the Deep Learning code using one of the most relevant framework, Python, and one of the best-known libraries: \TBD{Tensorflow, Keras, or Pytorch (To Be chosen in further stages)} .
    \item Use an agile methodology: SCRUM. This methodology should be used in the project. We will use the Zenhub tool of Github as a helper in the project management.
\end{itemize}

These next objectives will be addressed if the primary ones are reached. We could see this aims like a extra for the project. If we achieved good performance in this task, we would research about how to apply this solution to disease detection or data augmentation.

\begin{itemize}
    \item Build a semi-supervised autoencoder.
    \item Build a tumor detection system (based on supervised learning or based in the output of the autoencoder \cite{pinaya2019}).
    \item Research GAN architectures for noise and artifact reduction.
\end{itemize}
