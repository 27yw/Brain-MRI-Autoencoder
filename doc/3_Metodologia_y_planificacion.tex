\chapter{Planning and Methodology}
\label{chapter:planning}

In this chapter, we are going to discuss the scheduling for the project and the methodology used in this one.

\section{Research plan}
In this section, we are making a time planning for our project. Planning a project is a very important feature, because we can manage the time properly and we can keep a realistic task-calendar. For this purpose, we are going to elaborate a \textbf{Gantt Diagram}. This diagram is a very common resource used in project management \cite{tfm_cunha}.

Our diagram is a weekly Gannt Diagram. It has 17 weeks ([mm/dd/yyyy]): 

\begin{itemize}
    \item Week 1: from 09/14/2020 to 09/20/2020
    \item Week 17: from 01/01/2020 to 01/10/2021
\end{itemize}

It is built by all the main tasks that a master's degree final project must have and some personalized ones for this project. So we will have 6 big phases derived from project submits.

\TBD{This diagram is up to the date of September 27, 2020. We show it in the next page.}

\clearpage

% documentation. This reproduces an example from Wikipedia:
% http://en.wikipedia.org/wiki/Gantt_chart
%


\definecolor{barblue}{RGB}{153,204,254}
\definecolor{groupblue}{RGB}{51,102,254}
\definecolor{linkred}{RGB}{165,0,33}
%\renewcommand\sfdefault{phv}
%\renewcommand\mddefault{mc}
%\renewcommand\bfdefault{bc}
\setganttlinklabel{s-s}{START-TO-START}
\setganttlinklabel{f-s}{FINISH-TO-START}
\setganttlinklabel{f-f}{FINISH-TO-FINISH}
\sffamily
\begin{ganttchart}[
    canvas/.append style={fill=none, draw=black!5, line width=.75pt},
    hgrid style/.style={draw=black!5, line width=.75pt},
    vgrid={*1{draw=black!5, line width=.75pt}},
    today=2,
    today rule/.style={
      draw=black!64,
      dash pattern=on 3.5pt off 4.5pt,
      line width=1.5pt
    },
    today label font=\footnotesize\bfseries,
    title/.style={draw=none, fill=none},
    title label font=\bfseries\footnotesize\color{black!70},
    title label node/.append style={below=7pt},
    include title in canvas=false,
    bar label font=\mdseries\footnotesize\color{black!70},
    bar label node/.append style={left=1.2cm},
    bar/.append style={draw=none, fill=black!63},
    bar incomplete/.append style={fill=barblue},
    bar progress label font=\mdseries\scriptsize\color{black!60},
    group incomplete/.append style={fill=groupblue},
    group left shift=0,
    group right shift=0,
    group height=.5,
    group peaks tip position=0,
    group label node/.append style={left=.2cm},
    group progress label font=\bfseries\scriptsize,
    link/.style={-latex, line width=1.5pt, linkred},
    link label font=\scriptsize\bfseries,
    link label node/.append style={below left=-2pt and 0pt}
  ]{1}{17}
  \gantttitle[
    title label node/.append style={below left=7pt and -3pt}
  ]{WEEKS:\quad1}{1}
  \gantttitlelist{2,...,17}{1} \\
  \ganttgroup[progress=100]{Project Selection}{1}{1} \\
  \ganttgroup[progress=99]{Scope and planning}{2}{2} \\
  \ganttbar[
    progress=99,
    name=WBS1A
  ]{Title, Keywords and Abstract}{2}{2} \\
  \ganttbar[
    progress=95,
    name=tit
  ]{Overview, relevance and aims}{2}{2} \\
  \ganttbar[
    progress=95,
    name=over
  ]{Methodology and planning}{2}{2} \\[grid]
  \ganttgroup[progress=5]{State of art research}{3}{5} \\
  \ganttbar[progress=10]{Search bibliography}{3}{5} \\
  \ganttbar[progress=0]{Search similar code projects}{4}{5} \\
  \ganttbar[progress=0]{Resume state of art, redefine aims}{4}{5}\\[grid]
  
  \ganttgroup[progress=0]{Project development}{6}{14} \\
  \ganttbar[progress=0]{Data preparation}{6}{6} \\
  \ganttbar[progress=0]{Deep Learning Design}{6}{7} \\
  \ganttbar[progress=0, name=cod]{Deep Learning codification}{7}{9}\\
  \ganttbar[progress=0, name=exe]{Experiment execution}{10}{12}\\
  \ganttbar[progress=0]{Discuss results, improvements}{12}{13}\\
  \ganttbar[progress=0]{Update documentation}{14}{14}\\[grid]
  
  \ganttgroup[progress=0]{Documentation}{15}{16} \\[grid]
  
  \ganttgroup[progress=0]{Presentation and defence}{17}{17} \\
  \ganttlink[link type=f-s]{cod}{exe}
\end{ganttchart}
\rmfamily


\clearpage


\section{Methodology}

In this section we must choose a common academic Data Mining development methodology, in which there are described the phases, tasks and its relationships.

The description and nature of the project are very helpful at this point because the methodology used in the project will depend on the nature of it. The main characteristic of this project is its research-oriented purpose, so we can label the project as an \textbf{academic research project}. Nevertheless, the main objective of this research is to develop a software component (a Deep Convolutional Autoencoder). We can also describe the project as a \textbf{software project}. In addition, the project is located in the Machine Learning and Deep Learning areas. These areas are very related to Maths, Statistics, and Computer Science. In all of these fields, the aim is to analyze data in a quantitative way. We analyze how the variables are related, how the autoencoder performance with a concrete measure, how it trains getting concrete metrics (how it learns, time, overfitting...), etc. So our methodology should be \textbf{quantitative}. We will take a representative sample of brain MRI, we will train the autoencoder and inference the results to all the population. All this sample and inference techniques are addressed by the validation methods of Machine Learning (Train/test, Cross-validation to reduce bias, etc). 

So, due to the nature of the project, we have to apply a methodology for an \textbf{quantitative academic research project for data mining software development}. 

\begin{tcolorbox}
In a very summarized way, we will start researching the state of art, defining the problem, and proposing a model to solve the target problem. We will choose and prepare our data. Then we will develop the data mining software solution for this problem, evaluating each step. Finally, we will evaluate our model an get a conclusion for our hypothesis. Thus, \textbf{CRISP-DM methodology} embed all of these steps and it will be chosen as the project methodology.
\end{tcolorbox}

The methodology that best suits our project is \textbf{CRISP-DM} \cite{crisp}. The \textit{\textbf{CR}oss-\textbf{I}ndustry \textbf{S}tandard \textbf{P}rocess for \textbf{D}ata \textbf{M}ining} is a framework used for creating and deploying machine learning solutions. Moreover, research and quantitative tasks can be embedded  in the CRISP-DM phases (i.e. state of art research phase can fit into business understanding CRISP-DM phase and quantitative evaluation can fit into model evaluation).

As we know, agile methodologies are often used in software development. CRIPS-DM is neither an agile methodology nor a waterfall one. This methodology has clear stages, but the order of them is not strict and we could move forward and back whenever we need, in order to improve our data mining final model. In fact, this movement between phases is widely used. Also it has a iterative cycle, in which data, data preparation, modelling and evaluation are improved wit the previous iteration feedback.

Figure \ref{fig:figs/crisp_dm.png} shows the phase dependencies and order. As we can see, the straight lines define the dependencies between phases as in a classical methodology. Nevertheless, We can see the circle and the two-arrowed straight lines that show the flexibility and the agile similarity of CRISP-DM.


\imagen[0.5]{figs/crisp_dm.png}{CRISP-DM Cycle}

The phases of CRISP-DM \cite{crisp} are the following (\TBD{In later stages, references will be made to the specific sections of the document where the phases are carried out}):
\begin{description}
    \item[Business Understanding]: deep analysis of the business needs. In this phase we can establish an objective. In our case, we can research the state of art for Deep Convolutional Autoencoder for brain MRI and propose a model based on this research.
    \item[Data Understanding]: we should research the data sources as  \myurl{https://brain-development.org/ixi-dataset/}{IXI}, data quality and we should explore the data and its characteristics.
    \item[Data Preparation]: Data should be cleaned, filtered, selected and integrated if necessary. We could carry out tasks like preprocessing T1 weighted brain MRI or realize data augmentation. I will be explained in-depth in section \ref{section:datalifecycle}.
    \item[Modeling]: Specify the model to use and the architecture, parameters, etc. Maybe running several model architecture and hyper-parameter optimization to reach the most powerful model. So in can be an iterative process.
    \item[Evaluation]: We must evaluate models properly to get meaningful conclusions. There are many techniques of model evaluation and it should be made carefully.
    \item[Deployment - Publication]: As our main goal is academic research, this phase would be \textit{Publication}. The tasks are: review the project and generate the final report.
\end{description}