\chapter{Planning and Methodology}
\label{chapter:planning}

In this chapter, we are going to discuss the scheduling for the project and the methodology used in this one.

\section{Research plan}
In this section, we are making a time planning for our project. Planning a project is a very important feature, because we can manage the time properly and we can keep a realistic task-calendar. For this purpose, we are going to elaborate a \textbf{Gantt Diagram}. This diagram is a very common resource used in project management \cite{tfm_cunha}.

Our diagram is a weekly Gannt Diagram. It has 17 weeks ([mm/dd/yyyy]): 

\begin{itemize}
    \item Week 1: from 09/14/2020 to 09/20/2020
    \item Week 17: from 01/01/2020 to 01/10/2021
\end{itemize}

It is built by all the tasks that a master's degree final project must have and some personalized ones for this concrete project.

\TBD{This diagram is up to the date of September 27, 2020. We show it in the next page.}

\clearpage

% documentation. This reproduces an example from Wikipedia:
% http://en.wikipedia.org/wiki/Gantt_chart
%


\definecolor{barblue}{RGB}{153,204,254}
\definecolor{groupblue}{RGB}{51,102,254}
\definecolor{linkred}{RGB}{165,0,33}
%\renewcommand\sfdefault{phv}
%\renewcommand\mddefault{mc}
%\renewcommand\bfdefault{bc}
\setganttlinklabel{s-s}{START-TO-START}
\setganttlinklabel{f-s}{FINISH-TO-START}
\setganttlinklabel{f-f}{FINISH-TO-FINISH}
\sffamily
\begin{ganttchart}[
    canvas/.append style={fill=none, draw=black!5, line width=.75pt},
    hgrid style/.style={draw=black!5, line width=.75pt},
    vgrid={*1{draw=black!5, line width=.75pt}},
    today=2,
    today rule/.style={
      draw=black!64,
      dash pattern=on 3.5pt off 4.5pt,
      line width=1.5pt
    },
    today label font=\footnotesize\bfseries,
    title/.style={draw=none, fill=none},
    title label font=\bfseries\footnotesize\color{black!70},
    title label node/.append style={below=7pt},
    include title in canvas=false,
    bar label font=\mdseries\footnotesize\color{black!70},
    bar label node/.append style={left=1.2cm},
    bar/.append style={draw=none, fill=black!63},
    bar incomplete/.append style={fill=barblue},
    bar progress label font=\mdseries\scriptsize\color{black!60},
    group incomplete/.append style={fill=groupblue},
    group left shift=0,
    group right shift=0,
    group height=.5,
    group peaks tip position=0,
    group label node/.append style={left=.2cm},
    group progress label font=\bfseries\scriptsize,
    link/.style={-latex, line width=1.5pt, linkred},
    link label font=\scriptsize\bfseries,
    link label node/.append style={below left=-2pt and 0pt}
  ]{1}{17}
  \gantttitle[
    title label node/.append style={below left=7pt and -3pt}
  ]{WEEKS:\quad1}{1}
  \gantttitlelist{2,...,17}{1} \\
  \ganttgroup[progress=100]{Project Selection}{1}{1} \\
  \ganttgroup[progress=99]{Scope and planning}{2}{2} \\
  \ganttbar[
    progress=99,
    name=WBS1A
  ]{Title, Keywords and Abstract}{2}{2} \\
  \ganttbar[
    progress=95,
    name=tit
  ]{Overview, relevance and aims}{2}{2} \\
  \ganttbar[
    progress=95,
    name=over
  ]{Methodology and planning}{2}{2} \\[grid]
  \ganttgroup[progress=5]{State of art research}{3}{5} \\
  \ganttbar[progress=10]{Search bibliography}{3}{5} \\
  \ganttbar[progress=0]{Search similar code projects}{4}{5} \\
  \ganttbar[progress=0]{Resume state of art, redefine aims}{4}{5}\\[grid]
  
  \ganttgroup[progress=0]{Project development}{6}{14} \\
  \ganttbar[progress=0]{Data preparation}{6}{6} \\
  \ganttbar[progress=0]{Deep Learning Design}{6}{7} \\
  \ganttbar[progress=0, name=cod]{Deep Learning codification}{7}{9}\\
  \ganttbar[progress=0, name=exe]{Experiment execution}{10}{12}\\
  \ganttbar[progress=0]{Discuss results, improvements}{12}{13}\\
  \ganttbar[progress=0]{Update documentation}{14}{14}\\[grid]
  
  \ganttgroup[progress=0]{Documentation}{15}{16} \\[grid]
  
  \ganttgroup[progress=0]{Presentation and defence}{17}{17} \\
  \ganttlink[link type=f-s]{cod}{exe}
\end{ganttchart}
\rmfamily


\clearpage


\section{Methodology}

\TBD{To Be Done.}

\textbf{Estrategia de investigación: En este apartado se deben describir la estrategia de investigación a seguir (consultar el libro de Oates), así como las técnicas de generación de datos (cuantitativas y/o cualitativas) y las herramientas a utilizar (p.ej. MINITAB, ATLAS.TI, SPSS, etc.).}
\textbf{En cuanto a la descripción de la metodología de trabajo o investigación aplicada y conceptos clave: indicar cuáles son las posibles estrategias para llevar a cabo el trabajo e indicar cuál fue la estrategia escogida (p.ej.: desarrollar un producto nuevo, adaptar un producto existente, etc.). Valorar por qué esta es la estrategia más apropiada para conseguir los objetivos propuestos. En cuanto a la descripción general del proceso de trabajo/desarrollo realizado: describir las posibles metodologías de investigación (por ejemplo, encuestas, entrevistas), metodologías de desarrollo (por ejemplo, cascada, creación de prototipo, programación ágil), recursos, etc. utilizados para abordar el proyecto.}

La metodología debe incluir los pasos,
etapas o fases que se seguirán en el trabajo para alcanzar los objetivos propuestos
y validar (en su caso) las hipótesis planteadas. 

las técnicas, los métodos, las estrategias, las medidas, las herramientas, los
recursos, etc., empleados en el trabajo. Estos, por supuesto, variarán en función
del ámbito especializado en el que se enmarque el trabajo académico.

Utilizaremos metodologias de ciencia de datos: explicar metodologia de ciencia de datos y ciclo de vida. como analizar diseñar implementar y testear.

Contar las metodologias de desarrolo de algoritmos de DL en medicina. Ciclios, graficos de desarrollo: obtener datos- desarrrolar modelo- mejorar modelo, etc etc.

Contar etapas del desarrollo, como valorar el modelo, como gestionarle, etc

Benchmark

Para la gestion usar ciclos ágile.